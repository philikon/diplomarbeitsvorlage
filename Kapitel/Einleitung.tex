\chapter{Einleitung}

Die Verwendung der \texttt{subfigure}-Umgebung sei hier kurz anhand Abbildung~\ref{fig:unilogos} veranschaulicht. Auch zu Tabellen sei ein kleines Beispiel gegeben (siehe Tabelle~\ref{tab:beispiel}).

\begin{figure}
  \subfigure[Logo in blau.]{\includegraphics[width=0.47\textwidth]{Bilder/TU_Logo_blau.eps}}\hfill
  \subfigure[Logo in schwarz.]{\includegraphics[width=0.47\textwidth]{Bilder/TU_Logo_sw}}  
  \caption[Logos der TU Dresden]{Logos der Technischen Universit�t Dresden in zwei verschiedenen Farben.}
  \label{fig:unilogos}
\end{figure}

\begin{table}
  \caption[Beispieltabelle]{Beispieltabelle zur Demonstration von \texttt{multirow}, \texttt{multicolumn} und \texttt{tabularx}.}
  \centering
  \begin{tabularx}{0.9\textwidth}{rlrX}
  \hline
  \multicolumn{4}{l}{Messbedingungen} \\	
  Nr. & Richtung & Messzeit & Temperatur \\
  \hline		
  1   & hoch   & 1~min & \multirow{6}*{Es wurden jeweils 150\,{\degreecelsius} eingestellt.} \\
  2   & runter & 2~min \\
  3   & oben   & 3~min \\
  4   & unten  & 4~min \\
  5   & links  & 5~min \\
  6   & rechts & 6~min \\
  \hline
  \end{tabularx}
  \label{tab:beispiel}
\end{table}

Beispiele f�r Zitate: Ein Artikel \cite{LinSlo}, zwei B�cher \cite{kittel,raeuber} sowie ein Patent \cite{p-sampayan}, aber auch eine Homepage \cite{estar} wurden verwendet. Bei B�chern kann man gerne auch die Seite mit angeben \cite[S.~1]{kittel}.

Auf der ersten Seite eines jeden Kapitels wird �brigens immer die Kopfzeile weggelassen und es erscheint nur die Seitennummer unten.