%%%%%%%%%%%%%%%%%%%%%%%%%%%%%%%%%%%%%%%%%%%%%%%%%%%%%%%%%%%%%%%%%%%%%%%%%%%%
% Diplomarbeitsvorlage von Hartmut St�cker, Stand 25.01.2008
% Modifiziert von Philipp v. Weitershausen im April 2009


%%%%%%%%%%%%%%%%%%%%%%%%%%%%%%%%%%%%%%%%%%%%%%%%%%%%%%%%%%%%%%%%%%%%%%%%%%%%
% Optionen f�r die Dokumentklasse scrbook aus dem KOMA-Skript:
% - 12pt: Schriftgr��e 12 pt
% - twoside: Doppelseitiger Ausdruck
% - BCOR20mm: Bindekorrektur 20mm
% - DIV18: �u�erer Rand ca. 20mm (damit innerer Rand + BCOR = 30mm)
% - headinclude: Festlegung der Seitenr�nder unter Einbeziehung der Kopfzeile
% - liststotoc: Abbildungs- und Tabellenverzeichnis ins Inhaltsverzeichnis
% - bibtotoc: Literaturverzeichnis ins Inhaltsverzeichnis aufnehmen
% - tablecaptionabove: Tabellenbeschriftung �ber der Tabelle
% - ngerman: Neu-deutsche Silbentrennung
% - dvipdfm: Angabe des verwendeten Latex-Compilers (wichtig f�r hyperref)

\documentclass[12pt,twoside,BCOR20mm,DIV18,headinclude,headsepline,liststotoc,bibtotoc,tablecaptionabove,ngerman,dvipdfm]{scrbook}

% Laden der ben�tigten Pakete
\usepackage[dvipdfm]{graphicx}    % Grafiken einf�gen
\usepackage{subfigure}            % Teilbilder erzeugen
\usepackage{amsmath}              % Mathematikbefehle
\usepackage{amssymb}              % Unter anderem sch�ne Mengensymbole
\usepackage[amssymb]{SIunits}     % Einheiten per Befehl einf�gen
\usepackage{multirow}             % Tabellenzeilen verbinden
\usepackage{tabularx}             % Tabellen mit fester Breite

% Benutze Times as Serif-Schrift, Helvetica als Sans-Serif-Schrift.
% Ist angenehmer zu lesen als Computer Modern (LaTeXs Standard-
% Serif-Schrift) und sorgt nebenbei f�r kompaktere Postscript-
% bzw. PDF-Dateien.
\usepackage{times}

% PDF-Dokumenteigenschaften setzen, PDF-Lesezeichen erzeugen, Verweise anklickbar machen
\usepackage[pdftitle={Diplomarbeit}, pdfauthor={Vorname Name}, bookmarks=true, pdfborder={0 0 0}]{hyperref}
% Verweise nicht auf Bildunterschrift, sondern auf Bild selbst
\usepackage[all]{hypcap}

% Bildunterschrift 0.8 cm einr�cken, Gr��e small, Abbildung wird fett geschrieben
\usepackage[margin=0.8cm, font=small, labelfont=bf]{caption}

% Deutsche Zeichen- und Sprachpakete
% Unter Windows ist 'latin1' durch 'ansinew' zu ersetzen
\usepackage[latin1]{inputenc}
\usepackage[T1]{fontenc}
\usepackage[ngerman]{babel}

% Keine Einr�ckung am Absatzanfang
\setlength{\parindent}{0mm}

% 1,5 facher Zeilenabstand
\usepackage{setspace}
\onehalfspacing

% Formel-, Bilder- und Tabellennummer mit section.nummer
\numberwithin{equation}{section}
\numberwithin{figure}{section}
\numberwithin{table}{section}

% Kopfzeile mit Kapitelname auf der linken, Abschnittsname auf der
% rechten Seite.  Fu�zeile mit Seitennummer.
\usepackage{scrpage2}
\pagestyle{scrheadings}
\automark[section]{chapter}


%%%%%%%%%%%%%%%%%%%%%%%%%%%%%%%%%%%%%%%%%%%%%%%%%%%%%%%%%%%%%%%%%%%%%%%%%%%%
% Anfang des Dokuments

\begin{document}

% Seiten bis zum Beginn des eigentlichen Textes mit r�mischen Seitenzahlen in der Fu�zeile
\frontmatter
\pagestyle{plain}


%%%%%%%%%%%%%%%%%%%%%%%%%%%%%%%%%%%%%%%%%%%%%%%%%%%%%%%%%%%%%%%%%%%%%%%%%%%%
% Titelseite

\thispagestyle{empty}

\vspace{-5em}
\begin{flushleft}\includegraphics[angle=0,width=60mm]{Bilder/TU_Logo_sw.eps}\par\end{flushleft}
\vspace{-3em}
\begin{center}\rule{\textwidth}{0.1ex}\par\end{center}
\vspace{-4em}
\begin{center}\rule{\textwidth}{0.1ex}\par\end{center}

\vfill

\begin{center}\textbf{\Huge Hier kommt}\par\end{center}
\begin{center}\textbf{\Huge der Titel}\par\end{center}
\begin{center}\textbf{\Huge hin}\par\end{center}

\vfill

\begin{center}
{\large Diplomarbeit}\\
{\large zur Erlangung des wissenschaftlichen Grades}\\
{\large Diplom-Physiker}\par
\end{center}

\begin{center}vorgelegt von\par\end{center}
\begin{center}{\large Vorname Name}\\geboren am xx.yy.zzzz in Ort\par\end{center}

\vspace{13mm}

\begin{center}
{\large Institut f�r ...}\\
{\large der Technischen Universit�t Dresden}\\
{\large 200x}\par
\end{center}


%%%%%%%%%%%%%%%%%%%%%%%%%%%%%%%%%%%%%%%%%%%%%%%%%%%%%%%%%%%%%%%%%%%%%%%%%%%%%
% Gutachter-Seite

\newpage
\thispagestyle{empty}

\ \vfill

\begin{tabular}{l}
Eingereicht am xx.yy.zzzz \\
\end{tabular}

\vspace{1cm}

\begin{tabular}{ll}
1. Gutachter: & ...\\
2. Gutachter: & ...\\
\end{tabular}


%%%%%%%%%%%%%%%%%%%%%%%%%%%%%%%%%%%%%%%%%%%%%%%%%%%%%%%%%%%%%%%%%%%%%%%%%%%%%
% Zusammenfassung/Abstract-Seite

\newpage
\thispagestyle{empty}

\section*{\centering{Kurzdarstellung}}

Abstract auf Deutsch...

\vspace{3mm}

\section*{\centering{Abstract}}

Abstract auf Englisch...


%%%%%%%%%%%%%%%%%%%%%%%%%%%%%%%%%%%%%%%%%%%%%%%%%%%%%%%%%%%%%%%%%%%%%%%%%%%%%
% Inhaltsverzeichnis, Abbildungsverzeichnis, Tabellenverzeichnis

\tableofcontents
\listoffigures
\listoftables


%%%%%%%%%%%%%%%%%%%%%%%%%%%%%%%%%%%%%%%%%%%%%%%%%%%%%%%%%%%%%%%%%%%%%%%%%%%%%
% Haupttext

\mainmatter
\pagestyle{scrheadings}         % Oben definierte Kopf- und Fu�zeile verwenden

\chapter{Einleitung}

Die Verwendung der \texttt{subfigure}-Umgebung sei hier kurz anhand Abbildung~\ref{fig:unilogos} veranschaulicht. Auch zu Tabellen sei ein kleines Beispiel gegeben (siehe Tabelle~\ref{tab:beispiel}).

\begin{figure}
  \subfigure[Logo in blau.]{\includegraphics[width=0.47\textwidth]{Bilder/TU_Logo_blau.eps}}\hfill
  \subfigure[Logo in schwarz.]{\includegraphics[width=0.47\textwidth]{Bilder/TU_Logo_sw}}  
  \caption[Logos der TU Dresden]{Logos der Technischen Universit�t Dresden in zwei verschiedenen Farben.}
  \label{fig:unilogos}
\end{figure}

\begin{table}
  \caption[Beispieltabelle]{Beispieltabelle zur Demonstration von \texttt{multirow}, \texttt{multicolumn} und \texttt{tabularx}.}
  \centering
  \begin{tabularx}{0.9\textwidth}{rlrX}
  \hline
  \multicolumn{4}{l}{Messbedingungen} \\	
  Nr. & Richtung & Messzeit & Temperatur \\
  \hline		
  1   & hoch   & 1~min & \multirow{6}*{Es wurden jeweils 150\,{\degreecelsius} eingestellt.} \\
  2   & runter & 2~min \\
  3   & oben   & 3~min \\
  4   & unten  & 4~min \\
  5   & links  & 5~min \\
  6   & rechts & 6~min \\
  \hline
  \end{tabularx}
  \label{tab:beispiel}
\end{table}

Beispiele f�r Zitate: Ein Artikel \cite{LinSlo}, zwei B�cher \cite{kittel,raeuber} sowie ein Patent \cite{p-sampayan}, aber auch eine Homepage \cite{estar} wurden verwendet. Bei B�chern kann man gerne auch die Seite mit angeben \cite[S.~1]{kittel}.

Auf der ersten Seite eines jeden Kapitels wird �brigens immer die Kopfzeile weggelassen und es erscheint nur die Seitennummer unten.    % Dateien ohne Endung .tex einf�gen!
\chapter{Grundlagen}

\section{Allgemeine Grundlagen}

Hier ein Feynman-Diagramm zu $(g-2)_\mu$ im MSSM:

\begin{center}
\begin{fmffile}{Feynman/QED+Chargino}
  \fmfset{thin}{.5pt}
  \fmfset{wiggly_len}{3mm}
  \fmfset{dash_len}{2.5mm}
  \fmfset{dot_size}{1thick}
  \fmfset{arrow_len}{2.5mm}

\begin{fmfgraph*}(80,60)
  \fmfkeep{qedcha-1a}
  \fmfleft{gamma}
  \fmfright{in,out}
  \fmf{plain}{out,va,v1,vb,vertex,vc,v2,vd,in}
  \fmf{photon}{vertex,gamma}
  \fmf{photon,tension=0}{v1,v2}
  \fmffreeze
  \fmf{dashes,right}{va,vb}
  \fmfdot{vertex,v1,v2,va,vb}
\end{fmfgraph*}
\end{fmffile}
\end{center}

Falls Feynman-Diagramme in Formel vorkommen sollen, empfiehlt es sich,
sie vorher in einer {\ttfamily sbox} zu definieren und dann sp�ter mit
{\ttfamily $\backslash$fmfreuse} aufzurufen:

% Use \sbox to prevent the diagrams from being displayed. We can now
% refer to them indidivually by their id (\fmfkeep)
\newsavebox{\charginodiagrams}
\sbox{\charginodiagrams}{
\begin{fmffile}{Feynman/Nichtplanar}
  \fmfset{thin}{.5pt}
  \fmfset{wiggly_len}{3mm}
  \fmfset{dash_len}{2.5mm}
  \fmfset{dot_size}{1thick}
  \fmfset{arrow_len}{2.5mm}

\begin{fmfgraph*}(80,60)
  \fmfkeep{qedcha-4a}
  \fmfleft{gamma}
  \fmfright{in,out}
  \fmf{plain}{out,va,v1,vertex,v2,vb,in}
  \fmf{photon}{vertex,gamma}
  \fmffreeze
  \fmf{dashes}{v1,vb}
  \fmf{photon}{v2,va}
  \fmfdot{vertex,v1,v2,va,vb}
\end{fmfgraph*}

\begin{fmfgraph*}(80,60)
  \fmfkeep{qedcha-4b}
  \fmfleft{gamma}
  \fmfright{in,out}
  \fmf{plain}{out,va,v1,vertex,v2,vb,in}
  \fmf{photon}{vertex,gamma}
  \fmffreeze
  \fmf{dashes}{v2,va}
  \fmf{photon}{v1,vb}
  \fmfdot{vertex,v1,v2,va,vb}
\end{fmfgraph*}
\end{fmffile}
}

\begin{equation}
  \fmfreuse{qedcha-4a} + \fmfreuse{qedcha-4b} = \text{endlich}
\end{equation}

Sch�ner sieht's vertikal zentriert aus.  Das geht mit {\ttfamily
  $\backslash$vcenter}, wenn man das Diagramm in eine {\ttfamily
  $\backslash$hbox} verpackt:

\newcommand{\fmfvcenter}[1]{\vcenter{\hbox{\fmfreuse{#1}}}}

\begin{equation}
  \fmfvcenter{qedcha-4a} + \fmfvcenter{qedcha-4b} = \text{endlich}
\end{equation}

Die modifizierte feynMF-Version erlaubt auch das Malen h�bscher
Countertermdiagramme mit dem neuen Befehl {\ttfamily $\backslash$fmfct}:

\newsavebox{\qedcounterterms}
\sbox{\qedcounterterms}{
\begin{fmffile}{Feynman/Counterterme}
  \fmfset{thin}{.5pt}
  \fmfset{wiggly_len}{3mm}
  \fmfset{dash_len}{2.5mm}
  \fmfset{dot_size}{1thick}
  \fmfset{arrow_len}{2.5mm}

\begin{fmfgraph*}(80,60)
  \fmfkeep{qedct-1a}
  \fmfleft{gamma}
  \fmfright{in,out}
  \fmf{plain}{out,v1,va,vertex,vb,v2,in}
  \fmf{photon}{vertex,gamma}
  \fmf{photon,tension=0}{v1,v2}
  \fmfdot{vertex,v2}
  \fmfct{v1}
\end{fmfgraph*}
\end{fmffile}
}

\begin{equation}
  \fmfvcenter{qedcha-1a} + \fmfvcenter{qedct-1a} = \text{endlich}
\end{equation}


\subsection{Erster Unterabschnitt}

Erster Unterabschnitt...

\subsection{Zweiter Unterabschnitt}

Zweiter Unterabschnitt...

\section{Etwas speziellere Grundlagen}

Etwas speziellere Grundlagen...

\section{Spezielle Grundlagen}

Spezielle Grundlagen...

\chapter{Messverfahren}
\label{sec:Aufbau}

\section{Erstes Messverfahren}

Erstes Messverfahren...

\section{Zweites Messverfahren}

Zweites Messverfahren...
\chapter{Messwerte}
\label{sec:Werte}

\section{Messwerte des ersten Teilversuchs}

Messwerte des ersten Teilversuchs...

\section{Messwerte des zweiten Teilversuchs}

Messwerte des zweiten Teilversuchs...
\chapter{Diskussion der Ergebnisse}
\label{sec:disk}

\section{Erster Diskussionsabschnitt}

Diskussion der Ergebnisse...

\section{Zweiter Diskussionsabschnitt}

Diskussion der Ergebnisse...
\include{Kapitel/Zusammenfassung}


%%%%%%%%%%%%%%%%%%%%%%%%%%%%%%%%%%%%%%%%%%%%%%%%%%%%%%%%%%%%%%%%%%%%%%%%%%%%%
% Anhang

\appendix

\chapter{Anhang}

Sachen, die im Hauptteil st�ren w�rden...


%%%%%%%%%%%%%%%%%%%%%%%%%%%%%%%%%%%%%%%%%%%%%%%%%%%%%%%%%%%%%%%%%%%%%%%%%%%%%
% Literaturverzeichnis

\backmatter
\newpage
\pagestyle{plain}             % nur Nummerierung in der Fu�zeile

\bibliographystyle{alpha}     % Zitierstil: alpha = [Nam88]
\bibliography{Literatur}      % BibTeX-Datei name.bib ohne .bib hier einf�gen
%\nocite{*}                    % Listet alle Eintr�ge der Datei auf, wenn aktiv


%%%%%%%%%%%%%%%%%%%%%%%%%%%%%%%%%%%%%%%%%%%%%%%%%%%%%%%%%%%%%%%%%%%%%%%%%%%%%
% Danksagung

\newpage
\thispagestyle{empty}

\section*{\centering{Danksagung}}\bigskip

{\parindent 0mm

Allgemeine Dankesworte...\bigskip

Spezielle Dankesworte...\bigskip

Weitere Dankesworte...

}


%%%%%%%%%%%%%%%%%%%%%%%%%%%%%%%%%%%%%%%%%%%%%%%%%%%%%%%%%%%%%%%%%%%%%%%%%%%%%
% Erkl�rung

\newpage
\thispagestyle{empty}

\section*{\centering{Erkl�rung}}\bigskip

Hiermit versichere ich, dass ich die vorliegende Arbeit ohne unzul�ssige Hilfe Dritter und ohne Benutzung anderer als der angegebenen Hilfsmittel angefertigt habe. Die aus fremden Quellen direkt oder indirekt �bernommenen Gedanken sind als solche kenntlich gemacht. Die Arbeit wurde bisher weder im Inland noch im Ausland in gleicher oder �hnlicher Form einer anderen Pr�fungsbeh�rde vorgelegt.\\[2.3cm]
Vorname Name\\
Dresden, Monat 200x


\end{document}